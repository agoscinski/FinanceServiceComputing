% !Mode:: "TeX:UTF-8"
\documentclass[a4paper, 11pt]{article}
\usepackage{xeCJK}
\setCJKmainfont{AR PL UMing CN} 
\usepackage{comment} % enables the use of multi-line comments (\ifx \fi)
\usepackage{lipsum} %This package just generates Lorem Ipsum filler text.
\usepackage{fullpage} % changes the margin
\usepackage{hyperref}

\begin{document}
%Header-Make sure you update this information!!!!
\noindent

\large\textbf{Conclusion Report}
\hfill \textbf{Exchange Simulator Project / Team B} \\

\normalsize Course code: CS28011 \hfill Prof. Jian Cao \& Dr. James L. Mei\\

TA: Nengjun Zhu  \hfill Due Date: 2016/12/30 \\

Teammates:
%TODO all write your student number, check your name
Alexander Goscinski 116030990050

Ruth-Emely Pierau

Valentin Rothoft

Yelinsheng(查汗巴依尔·叶林生) 116033910057

Husein Sulianto



\section*{Problem Statement}
%Put your Problem statement here! Example of a Citation\cite[p.219]{Robotics}. Here's Another Citation\cite{Flueck}
%TODO Husein write requirements for the project

\section*{System Architecture}
\lipsum[2]

\section*{Key Techniques}

\subsection*{Order Matching}

\subsection*{Server}

\subsection*{Client}

\subsection*{GUI}

\begin{itemize}
  \item Use python's GUI library HtmlPy to create GUI using html, css, javascript.
  \item Use CSS library Bootstrap to create fancy front end page. 
  \item Use JS library Jquery to add logic.
\end{itemize}

=======
The project is mainly written in python. We used python 2.7. For some features like the matching algorithm the numpy library \cite{numpy} was used.
For the FIX communication we used the quickfix library. For the database we use MySQL and use the library MySQLdb as interface between python and MySQL.



\section*{Task Allocation Among Members}
Alexander Goscinski:
\begin{itemize}
  \item Implementation and maintenance of system architecture
  \item Manage tasks
\end{itemize}
Yelinsheng(查汗巴依尔·叶林生):

\begin{itemize}
  \item GUI implementation
  \item Some functions of client logic
  \item Some functions of server logic
  \item Implementation of some basic data classes
  \item Code testing
\end{itemize}
% to comment sections out, use the command \ifx and \fi. Use this technique when writing your pre lab. For example, to comment something out I would do:
%  \ifx
%   \begin{itemize}
%       \item item1
%       \item item2
%   \end{itemize}
%  \fi

\section*{Workload Allocations}
Total 100\% \\

\section*{Other}
\lipsum[7]

\section*{Conclusion}
The usage of the quickfix library with pythos is recommended because the port is not fitted to python code. It is mostly Java code with python grabbers.

\lipsum[6]


\section*{Attachments}
%Make sure to change these
Lab Notes, HelloWorld.ic, FooBar.ic
%\fi %comment me out

\begin{thebibliography}{9}
\bibitem{numpy} Developers, NumPy. "NumPy." NumPy Numpy. Scipy Developers (2013) \url{https://docs.scipy.org/doc/numpy-dev/numpy-ref.pdf}.
\bibitem{Robotics} Fred G. Martin \emph{Robotics Explorations: A Hands-On Introduction to Engineering}. New Jersey: Prentice Hall.
\bibitem{Flueck}  Flueck, Alexander J. 2005. \emph{ECE 100}[online]. Chicago: Illinois Institute of Technology, Electrical and Computer Engineering Department, 2005 [cited 30 August 2005]. Available from World Wide Web: (http://www.ece.iit.edu/~flueck/ece100).
\end{thebibliography}

\end{document}
